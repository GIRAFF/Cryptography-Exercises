\documentclass[oneside, final, 12pt]{extarticle}
\usepackage[utf8]{inputenc}
\usepackage[russian]{babel}
\usepackage{vmargin}
\usepackage{listings}
\usepackage{amsmath}
\usepackage{amssymb}
%\usepackage{graphicx}
%\usepackage{ucs}
\setpapersize{A4}
\setmarginsrb{2cm}{2cm}{2cm}{2cm}{0pt}{0mm}{0pt}{13mm}
\usepackage{indentfirst}		%красная строка
%\usepackage{color}
\sloppy

\begin{document}
\begin{titlepage}
	\begin{centering}
		\textsc{Министерство образования и науки Российской Федерации}\\
		\textsc{Новосибирский государственный технический университет}\\
		\textsc{Кафедра теоритической и прикладной информатики}\\
	\end{centering}
	%\centerline{\hfill\hrulefill\hrulefill\hfill}
	\vfill
	\vfill
	\vfill
	\Large
	\centerline{Лабораторная работа №3}
	\centerline{по дисциплине "<Основы теории информации и криптографии">}
	\centerline{\bfПомехоустойчивое кодирование}
	\normalsize
	\vfill
	\vfill
	\vfill
	\begin{flushleft}
		\begin{minipage}{0.3\textwidth}
			\begin{tabular}{l l}
				Факультет: & ПМИ\\
				Группа: & ПМИ-41\\
				Студент: & Кислицын И. О.\\
				Преподаватель: & Гультяева Т. А. 
			\end{tabular}
		\end{minipage}
	\end{flushleft}
	\vfill
	\vfill
	\begin{centering}
		Новосибирск\\
		2016\\
	\end{centering}
\end{titlepage}
\setcounter{page}{2}
\lstset{
	breaklines=\true,
	%frame=single,
	basicstyle=\footnotesize\ttfamily,
	tabsize=2,
	showspaces=\false,
	breaklines=\true,
	breakatwhitespace=\true,
	%escapeinside={[}{]},
	%inputencoding=utf8x,
	extendedchars=\true,
	keepspaces=\true,
	language=Haskell
}
\section{Цель работы}
Освоить основные алгоритмы помехоустойчивого кодирования.

\section{Задание}
\begin{enumerate}
	\item Реализовать приложение, кодирующее и декодирующее заданную последовательность символов по алгоритму Шеннона-Фано для случая равновероятного символов с проверкой на чётность.
	\item Реализовать приложение, кодирующее и декодирующее заданную последовательность символов по алгоритму Хэмминга для матрицы:
		\[ H_{(9,5)}\,=\,\left(
			\begin{matrix}
				0 & 0 & 0 & 0 & 0 & 0 & 0 & 1 & 1 \\
				0 & 0 & 0 & 1 & 1 & 1 & 1 & 0 & 1 \\
				0 & 1 & 1 & 0 & 0 & 1 & 1 & 0 & 1 \\
				1 & 0 & 1 & 0 & 1 & 0 & 1 & 0 & 0 \\
			\end{matrix} \right)
		\].
\end{enumerate}

\section{Текст программы}
Для выполнения задания 1 понадобилось лишь модифицировать функции кодирования и декодирования в программе из лабораторной работы №2.

\lstset{caption=Изменения в Logic.hs}
\begin{lstlisting}[]
{- ... -}
	
{- get code of symbol from codeList -}
getCode :: [(String, [Int])] -> String -> [Int]
getCode alph c
	| null matches = []
	| otherwise =  snd . head $ matches
	where
		matches = (filter (\(s, _) -> s == c) alph)

{- encode message with given codeList -}
encode :: [(String, [Int])] -> [String] -> [Int]
encode alph msg = foldl (++) [] (
		map (\x -> x ++ [(foldl (+) 0 x) `mod` 2])
		(map (getCode alph) msg)
		)

{- decode message with given tree -}
decode :: Tree -> [Int] -> [String]
decode t [] = []
decode t bs = fst dc : decode t (snd dc)
	where
		dc = decodeChar t bs 0

{- get char from given code -}
decodeChar :: Tree -> [Int] -> Int -> (String, [Int])
decodeChar (NodeInner(l, r)) (b:bs) par =
	if b == 1
	then decodeChar l bs (par+1)
	else decodeChar r bs par
decodeChar (Node(Symbol(c, _))) (b:bs) par =
	if b == par `mod` 2
	then (c, bs)
	else ("NUL",bs)

\end{lstlisting}

\lstset{caption=Реализация пользовательского интерфейса задания 2}
\lstinputlisting[language=Haskell]{HamMain.hs}

\lstset{caption=Реализация необходимых вычислений задания 2}
\lstinputlisting[language=Haskell]{Hamming.hs}

\section{Исследование 1}
Для всех тестов использовался алфавит \(\{0,1,2,3,4,5,6,7\}\), вероятность появления каждого символа равна \(0,1\).

\subsection*{Тестирование}
\begin{enumerate}
	\item \(0,6\;\rightarrow\;0,0,0,0,1,1,0,0\)
	\item \(0,0,0,0,1,1,0,0\;\rightarrow\;0,6\)
	\item \(0,0,0,0,1,0,0,0\;\rightarrow\;0,NUL\)
	\item \(0,0,0,1,1,0,0,0\;\rightarrow\;NUL,NUL\)
\end{enumerate}

Кодовое расстояние \(d_0\,=\,2\). Значит, мы можем обнаружить однократную ошибку.

\subsection*{Граница Хэмминга}
\[r\,=\,n\,-\,k\,\geqslant\,\log_2\sum_{i=0}^{q_{corr}} C_n^i\;\rightarrow\;1\,\geqslant\,0\]

\subsection*{Граница Плоткина}
\[d_0\,\leqslant\,n\frac{2^{k-1}}{2^k-1}\;\rightarrow\;2\,\leqslant\,\frac{16}{7}\]

\subsection*{Граница Варшамова-Гильберта}
\[2^r\,>\,\sum_{i=0}^{d_0-2}C_{n-1}^i\;\rightarrow\;2\,>\,0\]

\section{Исследование 2}
В программе используется алфавит \(\{0,1,...,31\}\).

\subsection*{Тестирование}
\begin{enumerate}
	\item \(12,5\;\rightarrow\;110011000,000001011\)
	\item \(110011000,000001011\;\rightarrow\;12,5\)
	\item \(11001[0]000,000001011\;\rightarrow\;12,5\)
	\item \(11001[0]000,[1]00001011\;\rightarrow\;12,5\)
\end{enumerate}

Кодовое расстояние \(d_0\,=\,3\). Значит, мы можем исправить однократную ошибку.

\subsection*{Граница Хэмминга}
\[r\,=\,n\,-\,k\,\geqslant\,\log_2\sum_{i=0}^{q_{corr}} C_n^i\;\rightarrow\;4\,\geqslant\,\log_2 9\]

\subsection*{Граница Плоткина}
\[d_0\,\leqslant\,n\frac{2^{k-1}}{2^k-1}\;\rightarrow\;3\,\leqslant\,4,65\]

\subsection*{Граница Варшамова-Гильберта}
\[2^r\,>\,\sum_{i=0}^{d_0-2}C_{n-1}^i\;\rightarrow\;16\,>\,8\]

\end{document}
