\documentclass[oneside, final, 12pt]{extarticle}
\usepackage[utf8]{inputenc}
\usepackage[russian]{babel}
\usepackage{vmargin}
\usepackage{listings}
\usepackage{amsmath}
\usepackage{amssymb}
%\usepackage{graphicx}
%\usepackage{ucs}
\setpapersize{A4}
\setmarginsrb{2cm}{2cm}{2cm}{2cm}{0pt}{0mm}{0pt}{13mm}
\usepackage{indentfirst}		%красная строка
%\usepackage{color}
\sloppy

\begin{document}
\begin{titlepage}
	\begin{centering}
		\textsc{Министерство образования и науки Российской Федерации}\\
		\textsc{Новосибирский государственный технический университет}\\
		\textsc{Кафедра теоритической и прикладной информатики}\\
	\end{centering}
	%\centerline{\hfill\hrulefill\hrulefill\hfill}
	\vfill
	\vfill
	\vfill
	\Large
	\centerline{Лабораторная работа №5}
	\centerline{по дисциплине "<Основы теории информации и криптографии">}
	\centerline{\bfГенерирование равномерно распределённых}
	\centerline{\bfпсевдослучайных последовательностей}
	\normalsize
	\vfill
	\vfill
	\vfill
	\begin{flushleft}
		\begin{minipage}{0.3\textwidth}
			\begin{tabular}{l l}
				Факультет: & ПМИ\\
				Группа: & ПМИ-41\\
				Студент: & Кислицын И. О.\\
				Преподаватель: & Гультяева Т. А. 
			\end{tabular}
		\end{minipage}
	\end{flushleft}
	\vfill
	\vfill
	\begin{centering}
		Новосибирск\\
		2016\\
	\end{centering}
\end{titlepage}
\setcounter{page}{2}
\lstset{
	breaklines=\true,
	%frame=single,
	basicstyle=\footnotesize\ttfamily,
	tabsize=2,
	showspaces=\false,
	breaklines=\true,
	breakatwhitespace=\true,
	%escapeinside={[}{]},
	%inputencoding=utf8x,
	extendedchars=\true,
	keepspaces=\true,
	language=Haskell
}
\section{Цель работы}
Освоить основные алгоритмы программной генерации равномерно распределенных псевдослучайных последовательностей.

\section{Задание}
Написать программу, генерирующую псевдослучайную последовательность с помощью линейного конгруэнтного генератора по заданным параметрам.

\section{Текст программы}

\lstset{caption=Пользовательский интерфейс}
\lstinputlisting[language=Haskell]{Main.hs}

\lstset{caption=Вычисления}
\lstinputlisting[language=Haskell]{Logic.hs}

\section{Исследования}
Найбольший период был достигут при \(a=1,\,c=4\). Полученная последовательность:

\begin{verbatim}
[4,8,12,1,5,9,13,2,6,10,14,3,7,11,0]
\end{verbatim}

По критерию Пирсона эта последовательность считается равномерно распределённой:

\[S^*=N\sum_{i=1}^K\frac{(\frac{v_i}{N}-P_i)^2}{P_i}\,=\,0\,<\,5,991\]
где \(N=15,\,K=3,\,P_i=\frac{1}{3}\,\forall i\).

\end{document}
