\documentclass[oneside, final, 12pt]{extarticle}
\usepackage[utf8]{inputenc}
\usepackage[russian]{babel}
\usepackage{vmargin}
\usepackage{listings}
\usepackage{amsmath}
\usepackage{amssymb}
%\usepackage{graphicx}
%\usepackage{ucs}
\setpapersize{A4}
\setmarginsrb{2cm}{2cm}{2cm}{2cm}{0pt}{0mm}{0pt}{13mm}
\usepackage{indentfirst}		%красная строка
%\usepackage{color}
\sloppy

\begin{document}
\begin{titlepage}
	\begin{centering}
		\textsc{Министерство образования и науки Российской Федерации}\\
		\textsc{Новосибирский государственный технический университет}\\
		\textsc{Кафедра теоритической и прикладной информатики}\\
	\end{centering}
	%\centerline{\hfill\hrulefill\hrulefill\hfill}
	\vfill
	\vfill
	\vfill
	\Large
	\centerline{Лабораторная работа №1}
	\centerline{по дисциплине "<Основы теории информации и криптографии">}
	\normalsize
	\vfill
	\vfill
	\vfill
	\begin{flushleft}
		\begin{minipage}{0.3\textwidth}
			\begin{tabular}{l l}
				Факультет: & ПМИ\\
				Группа: & ПМИ-41\\
				Студент: & Кислицын И. О.\\
				Преподаватель: & Гультяева Т. А. 
			\end{tabular}
		\end{minipage}
	\end{flushleft}
	\vfill
	\vfill
	\begin{centering}
		Новосибирск\\
		2016\\
	\end{centering}
\end{titlepage}
\setcounter{page}{2}
\lstset{
	breaklines=\true,
	%frame=single,
	basicstyle=\footnotesize\ttfamily,
	tabsize=2,
	showspaces=\false,
	breaklines=\true,
	breakatwhitespace=\true,
	escapeinside={[}{]},
	%inputencoding=utf8x,
	extendedchars=\true,
	keepspaces=\true
}
\section{Цель работы}
Приобретение навыка решения практических задач, отражающих основные свойства источников дискретных сообщений (ИДС).
\section{Ход работы}
\subsection{Задача 11}
Сообщения источника \(x_1,\,x_2,\,x_3,\,x_4\) для согласования с каналом кодируются в соответствии с таблицей.

~

\begin{tabular}{|c|c|c|c|c|} \hline
	Сообщение \(x_i\) & \(x_1\) & \(x_2\) & \(x_3\) & \(x_4\) \\ \hline
	\(p(x_i)\) & \(\frac{1}{2}\) & \(\frac{1}{4}\) & \(\frac{1}{8}\) & \(\frac{1}{8}\) \\ \hline
	Код & \(000\) & \(011\) & \(101\) & \(100\) \\ \hline
\end{tabular}

~

Пусть на вход кодера поступает сообщение \(x_3\). Вычислить дополнительную информацию об этом сообщении, которую содержит каждый последующий символ на выходе кодера.

\subsection*{Решение}
\[I(x_3;\text{'}1\text{'})=\log_2\frac{P(x_3|\text{'}1\text{'})}{P(x_3)};\]
\[P(x_3|\text{'}1\text{'})=\frac{P(x_3,\text{'}1\text{'})}{P(\text{'}1\text{'})}=\frac{P(x_3)P(\text{'}1\text{'}|x_3)}{\sum_{j} P(x_j)P(\text{'}1\text{'}|x_j)}=\frac{1}{2};\]
\[I(x_3;\text{'}1\text{'})=\log_2 4=2\;\text{бит};\]
\[I(x_3;\text{'}0\text{'}|\text{'}1\text{'})=\log_2\frac{P(x_3|\text{'}10\text{'})}{P(x_3|\text{'}1\text{'})}=\log_2 1=0;\]
\[P(x_3|\text{'}10\text{'})=\frac{P(x_3,\text{'}10\text{'})}{P(\text{'}10\text{'})}=\frac{P(x_3)P(\text{'}10\text{'}|x_3)}{\sum_{j} P(x_j)P(\text{'}10\text{'}|x_j)}=\frac{1}{2};\]
\[I(x_3;\text{'}1\text{'}|\text{'}10\text{'})=\log_2\frac{P(x_3|\text{'}101\text{'})}{P(x_3|\text{'}10\text{'})}=\log_2 2=1\;\text{бит};\]
\[P(x_3|\text{'}101\text{'})=1;\]
\[I(x_3)=3\;\text{бит};\]
\[I(x_3)\leftrightharpoons -\log_2P(x_3)=-\log_2\frac{1}{8}=3\;\text{бит}.\]

\subsection{Задача 13}
Источник вырабатывает ансамбль сообщений:
\[X=\left\{\begin{aligned} & x_1 \quad x_2 \quad x_3 \quad x_4 \\ & 0,2 \; 0,3 \; 0,4 \; 0,1 \end{aligned}\right\}\]
Символы в последовательности независимы. Вычислить энтропию источника и определить избыточность.

\subsection*{Решение}
\[H(\xi)=-\sum_{i=1}^{n}P(x_i)\log_2P(x_i)=1,846\;\text{бит};\]
\[K=1-\frac{H(\xi)}{H_{max}}=1-\frac{1,846}{2}=0,07.\]

\subsection{Задача 25}
Сообщение \(X\) есть стационарная последовательность независимых символов, имеющих ряд распределения:

~

\begin{tabular}{|c|c|c|c|c|c|} \hline
	\(i\) & 1 & 2 & 3 & 4 & 5 \\ \hline
	\(P(x_i)\) & 0,1 & 0,1 & 0,4 & 0,3 & 0,1 \\ \hline
\end{tabular}

~

Сигнал \(Y\) является последовательностью двоичных символов, связанных с сообщением \(X\) по следующему правилу:

\[x_1 \rightarrow 1,\,x_2 \rightarrow 0,\,x_3 \rightarrow 0,\,x_4 \rightarrow 1,\,x_5 \rightarrow 1.\]

Найти средние безусловную и условную энтропии сообщения \(X\) при условии, что было получено сообщение \(Y\).

\subsection*{Решение}
%\[H(X)=-\sum_{i=1}^{5}P(x_i)\log_2 P(x_i) = 2,046 \;\text{бит};\]
%\[y_1=1;\;y_2=0;\]
%\[P(y_1|x_1)=P(y_1|x_4)=P(y_1|x_5)=P(y_2|x_2)=P(y_2|x_3)=1;\]
%\[P(x,y)=P(x|y)P(y)=P(y|x)P(x);\;P(y_1)=P(y_2)=\frac{1}{2};\]
%\[H(X|Y)=-\sum_{j=1}^{n} \sum_{k=1}^{m} P(x_j,y_k) \log_2 P(x_j|y_k) = 1,0459 \; \text{бит}.\]
%\[H(Y|X)=-\sum_{j=1}^{n} \sum_{k=1}^{m} P(x_j,y_k) \log_2 P(y_k|x_j) = 0.\]
\end{document}
